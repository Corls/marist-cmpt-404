\documentclass[a4paper]{article}
\usepackage[letterpaper, margin=1in]{geometry} % page format
\usepackage{listings} % this package is for including code
\usepackage{graphicx} % this package is for including figures
\usepackage{amsmath}  % this package is for math and matrices
\usepackage{amsfonts} % this package is for math fonts
\usepackage{tikz} % for drawings
\usepackage{hyperref} % for urls

\title{Homework5}
\author{Corley Herman}
\date{11/14/16}

\begin{document}
\lstset{language=Python}

\maketitle

\noindent Question 1i:\\
The higher the n\_colors, the more detail appears in the resulting image. When the n\_colors are lower, the resulting image is blended a bit more. The reason behind this is because n\_colors is basically setting the amount of colors used to draw the image, so when it has a wider palette of colors, it gains the option to add more details through the slight variance of the colors.\\
\\
Question 1ii:\\
It could be used as a filter for cameras. There are filters that do something like this already, actually, if I'm not mistaken. That's all I can think of at the moment.\\
\\
Question 1iii:\\
I think the resulting picture chosen is funny because I find a certain humor in an image that can be drawn decently while only using 2 colors.\\
\\
Question 2A:\\
Ran this multiple times and then submitted them on the forms because I got ahead of myself. The results each time had eta at 0.1, but the neurons varied from 19 to 30 during these tests.\\
\\
Question 2C:\\
As I mentioned in the last one, I submitted multiple entries before realizing my mistake. For the google form, the only answer that had N=1000 was 95 neurons with an eta of 0.1\\
\\
\end{document}